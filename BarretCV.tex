\documentclass[oneside]{article}

% \usepackage{natbib,bibentry}
% \bibliographystyle{ieeetr}
% \nobibliography{BarretCV}

\setcounter{secnumdepth}{-1}

\usepackage[
 pdftex,
 bookmarks,
 colorlinks,
 pdfpagemode=UseOutlines,
 pdfauthor={Barret Schloerke},
 pdftitle={Curriculum Vitae.  Barret Schloerke},
 urlcolor=blue
]{hyperref}

% Tweak fonts and spacing
\usepackage[compact,small]{titlesec}

\textwidth = 6.5 in
\textheight = 9 in
\oddsidemargin = 0.0 in
\evensidemargin = 0.0 in
\topmargin = 0.0 in
\headheight = 0.0 in
\headsep = 0.0 in
\parskip = 0.2in
\parindent = 0.0in

%\usepackage[left=2.25cm,top=2.5cm,right=2.25cm,nohead]{geometry}
%\usepackage[spacing=true,kerning=true]{microtype}
\renewcommand\rmdefault{bch}
\linespread{1.07}
\setlength{\parindent}{0in}
\setlength{\parskip}{0.5em}

\usepackage{enumitem}
% \setlist{nolistsep}
\setlist{itemsep=.15em}
\setlist{topsep=0.1em}

\usepackage{ctable} % for \specialrule command

\newcommand{\thinline}{\specialrule{.02em}{0em}{0em}}


\begin{document}

\section*{Barret Schloerke 2015 }

(Permanent)\\
1711 Woodhaven Cir.\\
Ames, IA 50010 \\
515.290.0777 \\
\href{mailto:schloerke@gmail.com}{schloerke@gmail.com} \\
\url{http://www.github.com/schloerke/}

\section{Education}

\textbf{Purdue University} West Lafayette, Indiana

\begin{tabular}{p{1.5in}p{0.5in}p{1.75in}p{1.75in}}
PhD of Statistics & Advisors: & Dr. William Cleveland \& & Transferred: Spring 2013 \\
                  &           & Dr. Ryan Hafen & \\
\end{tabular}

\vspace{0.5em}

\textbf{Purdue University} West Lafayette, Indiana

\begin{tabular}{p{2.75in}p{0.75in}p{0.25in}p{1.75in}}
Masters of Mathematical Statistics & GPA: 3.69 & & Transferred: Spring 2013 \\
& & & Graduated Fall 2014  \\
\end{tabular}

\vspace{0.5em}

\textbf{Rice University} Houston, Texas

\begin{tabular}{p{1.5in}p{0.5in}p{1.75in}p{1.75in}}
PhD of Statistics & 
Advisor: & Dr. Hadley Wickham & 
Fall 2012
\end{tabular}

\vspace{0.5em}

\textbf{Iowa State University} Ames, Iowa

\begin{tabular}{p{2.75in}p{0.75in}p{0.25in}p{1.75in}}
Bach. of Sci. in Computer Engineering &
GPA: 3.77 & &
Graduated: December 2010
\end{tabular}

\vspace{1em}

\section{Honors and Awards}
\begin{tabular}{p{7cm}p{7cm}p{1.9cm}}
\textbf{Award} & \textbf{Description} & \textbf{Date} \\
\hline

Visiting Scientific Researcher &
Received full funding for \href{http://www.fields.utoronto.ca/programs/scientific/14-15/bigdata/visualization/}{Workshop on Visualization
for Big Data: Strategies and Principles} &
Spring 2015\\[0.5em]

\href{http://www.nsf.gov/grfp}{National Science Foundation Graduate \newline Research Fellowship} &
3 year fellowship grant recipient &
Spring 2012\\[0.5em]

\thinline

Best Dynamic / Interactive Visualization &
\href{http://datainsightsf.com/}{Data Insight (SF)} &
June 2011\\[0.5em]

\href{http://www.nsf.gov/grfp}{National Science Foundation Graduate \newline Research Fellowship} &
Honorable mention &
Spring 2011\\[0.5em]

\thinline

Magna Cum Laude &
Graduate with GPA greater than 3.70 &
Fall 2010\\[0.5em]

% Golden Keyboard &
% Managed a database for Greek Week &
% Spring 2010\\[0.5em]


% Golden Key International Honour Society &
% Being in the top 15\% of class &
% Inducted in \newline
% Fall 2008\\[0.5em]

Eta Kappa Nu Honor Society &
Being in top 25\% of class \newline
EE/CprE Honor Society &
Inducted in\newline
Fall 2008\newline\\[0.5em]

Dean's List & Achieving a 3.5 or above for a semester&
Fall 2006 -\newline Fall 2010\\[0.5em]

Engineering Merit Scholarship&
College of Engineering \newline Achieving at least a 3.50 GPA&
Fall 2006 - \newline Spring 2010\\[0.5em]

President's Award for Competitive Excellence &
For High School Excellence \newline 2nd in class rank &
Fall 2006 - \newline Spring 2010\\[0.5em]
\end{tabular}

\newpage
\section{Publications}
    
  \subsection{Journal Articles}
    

    \hangindent=2em
    Emerson, J., W. Green, B. Schloerke, J. Crowley, D. Cook, H. Hofmann, and H. Wickham ``\href{http://vita.had.co.nz/papers/gpp.pdf}{The Generalized Pairs Plot.}'' \emph{Journal of Computational and Graphical Statistics} 22.1 (2013). Print.
    
    \hangindent=2em
    Hofmann H., D. Cook, C. Kielion, B. Schloerke, J. Hobbs, A. Loy, L. Mosley, D. Rockoff, Y. Huang, D. Wrolstad, and T. Yin ``\href{http://amstat.tandfonline.com/doi/abs/10.1198/jcgs.2011.3de}{Delayed, Canceled, on Time, Boarding… Flying in the USA.}'' \emph{Journal of Computational and Graphical Statistics} 20.2 (2012). Print.

    %\bibentry{electionVis}
    \hangindent=2em
    Mosley, L., D. Cook, H. Hofmann, C. Kielion, and B. Schloerke. ``\href{http://chance.amstat.org/files/2010/12/Visually.pdf}{Monitoring the 2008 Election Visually.}'' \emph{Chance} 23.3 (2010). Print.
    
    %\bibentry{geoZooRJ}
    \hangindent=2em
    Schloerke, B., S. Bradley, H. Wickham, D. Cook, H. Hofmann. ``Escape from Boxland: Generating a Library of High-Dimensional Geometric Shapes.'' \emph{R Journal} In progress. Print.
    
    
  \subsection{Presentations}
  % \begin{itemize}
  
    \hangindent=2em
    Schloerke, B. ``ggplot2: displaying spatial and temporal data'' Spatial Statistics Seminar. Recitation, West Lafayette, IN. Oct. 2014. Speech.

    \hangindent=2em
    Schloerke, B. ``Reducing Working Environment Inefficiencies'' Graduate Statistics Seminar. Haas, West Lafayette, IN. Feb. 2014. Speech.



    \hangindent=2em
    Schloerke, B. ``helpr: Help for R.'' Working Group Meeting. Snedecor Hall, Ames, IA. Sept. 2010. Speech.
    % \bibentry{helprISU}
    % \item {\bf helpr:} Help for R
    %   \begin{itemize}
    %     \item Iowa State University Statistics Department
    %     \item Ames, IA September 2010
    %   \end{itemize}
  
    \hangindent=2em
    Schloerke, B. ``GGally: A Plot Matrix for All Variable Types.'' Joint Statistical Meeting 2010. Vancouver  Convention Center, Vancouver, BC Canada. Aug. 2010. Speech.
    % \bibentry{jsm2010}
    % \item {\bf GGally:} A plot matrix for all variable types
    %   \begin{itemize}
    %     \item Joint Statistical Meeting (Visualizing Data Graphics)
    %     \item Vancouver, BC Canada August 2010
    %   \end{itemize}
  
  
    \hangindent=2em
    Quart, G., B. Schloerke, and D. Gannon. ``Visualizing Large Data Sets: Housing Crisis.'' VIGRE. Rice  University, Houston, TX. June 2009. Speech.
    % \bibentry{vigre2009}
    % \item {\bf Visualizing Large Data Sets:} Housing Crisis
    %   \begin{itemize}
    %     \item Rice University VIGRE Summer Research Program
    %     \item Houston, TX June 2009
    %     \item Barret Schloerke, Dex Gannon, Gabi Quart
    %   \end{itemize}
  % \end{itemize}
  
  \subsection{Talks}
    \begin{itemize}
      \item Purdue working group (2014): ``git''
      \item Bay Area (2011): ``Sinartra: Simple Web Framework'', ``5 Tips for Running Node in Production'',
  ``Express: Node Router''
      \item ISU working group (2008-2010): ``Roxygen: Documentation for R'', ``helpr: Help for R'', and ``Electoral Towers''
    \end{itemize}
  

\section{Research}

\subsection{Purdue University}

\begin{itemize}
  \item{\bf trelliscopejs:} - Spring 2015
    \begin{itemize}      
      \item Ported original trelliscope R package to be separated from shiny
      \item Full visual implementation in javascript dramatically increases interaction time
      \item Utilized Ember javascript platform to produce a single webpage application with multiple URLS
    \end{itemize}


  \item{\bf Bitcoin:} DARPA XDATA - Summer 2013
    \begin{itemize}      
      \item Brought previous transaction exporter up to protocol
      \item Formatted escrow transactions to be matched easier
      \item Made ``None'' keys uniquely identifiable, rather than being treated the same
      \item Matched some ``None'' keys to valid output keys
      \item Updated User exporter to include all information available
    \end{itemize}
\end{itemize}


\subsection{Rice University VIGRE Program} Summer 2009
\begin{itemize}
  \item Performed exploratory analysis on the housing crisis
  \item Found, cleaned, and explored multiple data sets
  \item Created all outputs reproducible from scratch
  \item Supervised by Dr. Hadley Wickham
\end{itemize}

% VIGRE is a program sponsored by the National Science Foundation to carry out innovative educational programs in which research and education are integrated and in which undergraduates, graduate students, postdoctoral fellows, and faculty are mutually supportive.  During my stay at Rice University, I worked daily with two other undergraduate students, one graduate student, and our professor, Dr. Hadley Wickham.  
% 
% Our group project was to perform exploratory analysis on the housing crisis.  I found, cleaned, and explored multiple data sets.  The R programming language was used to clean the data.  To stay up-to-date with the other group members, Git, a repository language that evolved from SVN, was used constantly throughout the program.  One big requirement, which was followed during the project, was reproducibility.  Anyone could start  by retrieving our code and reproduce everything that we have already produced by sourcing the files we have made.  
% 
% Near the end of the internship, my undergraduate group gave a 20-minute speech on our findings, what we were currently working on, and where the group was heading in the future.  



\subsection{Novartis Pharmaceutical} Summer 2008
\begin{itemize}
  \item Created a time management tool using gantt charts to help management quickly display availability.
  \item Created a web-based tutorial consisting of examples of the most common graphics produced when using different layouts.  This allowed users to know exactly how everything was being displayed.  (white-box)
  \item Worked coherently with the top 15 statisticians within Novartis.
  \item Supervised by Dr. Anthony Rossini and Mentored by Dr. Francois Mercier.
\end{itemize}

% Modeling and Simulation group, Novartis AG. Basel, Switzerland. Invited by Anthony Rossini. Summer 2008. Development of graphical methods, tutorials for visualizing data, and a time management tool.
% 
% The web-based tutorial consisted of examples of most common graphics produced when using different layouts. It included new ways to display data as well. The time management tool produced outputs of the projects that people were involved in as reported in a .CSV file (exported .XLS file). The tool used an API to execute everything. It could allow the outputs to be split by project or similar criteria.
% 
% During the internship, I worked coherently with the top 15 statisticians within Novartis.  Each week we meet to learn about what each other was doing.


\subsection{ISU Statistics Research} Summer 2007 to Fall 2010
\begin{itemize}
  \item Supervised by Dr. Dianne Cook and Dr. Heike Hofmann
\end{itemize}
% Starting in the Summer of 2007, I worked under Dr. Dianne Cook and Dr. Heike Hofmann, ISU Statistics professors. Since then, I have explored different topics in graphics. Each project has touched on different areas. (Starting from most recent).

\begin{itemize}
  \item{\bf 2009 American Statistical Association Data Expo:} Delayed, Cancelled, On-Time, Boarding ... Flying in the USA - Spring 2009
    \begin{itemize}      
      \item Displayed airport delays caused by airport layout using multiple applications.
      \item Using Google's KML language, R and wget, we produced Google Earth images with the runways highlighted.
      \item Iowa State University working group finished 2nd in the \href{http://stat-computing.org/dataexpo/2009/posters/}{world competition}.
    \end{itemize}
    % \item{\bf 2009 ASA Data Expo:} Delayed, Cancelled, On-Time, Boarding ... Flying in the USA
    % 
    % The 2009 ASA Data Expo aim was to produce a graphical summary of important features.  Areas that I worked on dealt with which carriers were the best and why they were the best, which carriers performed the least amount of ``ghost flights'' (empty flights) and whether or not the layout of the airport was correlated with delays.  
    % 
    % Displaying the airport delays caused by airport layout used multiple applications.  Using Google's KML language, R and wget, Chris and I were able to produce Google Earth Images with the runways highlighted.  Following this, we produced images of the wind tendencies at each airport and matched them up with the corresponding airport runway direction.
    % 
    % The group finished 2nd in the \href{http://stat-computing.org/dataexpo/2009/posters/}{world competition}.
    
    

  \item{\bf Election Data:} Explore new ways to show the Electoral College for the 2008 Presidential Election - Fall 2008
  \begin{itemize}
    \item Created a custom cartogram of the United States with area proportional to the electoral vote of each state.
    \item Produced the `Electoral Vote Towers' (inspired by New York Times 2000 Presidential Race) that quickly portrayed the electoral vote status.
  \end{itemize}

  % This project focused on the electoral votes of each state. A cartogram of the United States with the area proportional to the electoral vote of each state and the `Electoral Vote Towers' were the two main graphics that I helped develop.
  % 
  % The cartogram shows voting pattern in a way that is more representative of the electoral votes while maintaining some geographic relations. The cartogram is block-constructed, coded, and imported into R. It demonstrates the states' area in proportion to its electoral votes. For example, Montana only has an area of three votes, while Rhode Island has an area of four votes even though it is much smaller geographically.  (Votes are determined by the number of representatives within the House and Senate, not by geographic area).
  % 
  % The `Electoral Vote Towers' was inspired by a New York Times display during the 2000 presidential election titled ?Building an Election Victory?. These displays show the sum of the electoral votes for each candidate. The shape comes from each state's percent difference (horizontal) and electoral votes (vertical). While the colors may be redundant, it still helps categorize the states' voting percentages. The goal of the candidates is to reach the 270 votes needed to win the presidency which corresponds to the towers building up to the 270th line. Each of the states that are still a toss-up have their relevant statistics reported on the candidate's side to which they are currently leaning.




\item{\bf DNA:} Viewing DNA sequence data - Fall 2008
\begin{itemize}
  \item Extension of the polytope ranking.
  \item Created combinations of DNA sequences with a length of four which were displayed in a grand tour.
\end{itemize}

% Dr. Michael Lawrence and Fred Hutchinson Cancer Research Center were working on a DNA data set and needed a way to view the distribution of DNA sequence samples relative to the possible combinations. The output needed to have a feasible way to tell trends while still showing as much data as possible.
% 
% The original goal was to make outputs for a sequence of 6 digits but was later reduced to 4 digits when some constraints were imposed. The project was an extension of the polytope ranking from Spring 2008. It developed proportionally sized spheres at each vertice/combination with points jittered on the surface of the sphere. This makes the most common sequences `pop' out at the viewer.

\item{\bf Polytope Ranking:} Visualizing ranked data - Spring 2008
\begin{itemize}
  \item Created a model that would connect a ranked data set.
  \item Added jittered data points to each combination with the distance relative to the amount of points.
\end{itemize}


\item{\bf FaceOff:} `Lines and Dots' in the fourth dimension - 2007
\begin{itemize}
  \item Game created to calibrate users into viewing and thinking in the fourth dimension.
  \item Game is a combination of  ``Connect 4'', ``Tic Tac Toe'', and ``Dots and Lines.''
\end{itemize}

% FaceOff is a game that is designed to help calibrate your eyes into viewing and mind into thinking in the fourth dimension. In 2004, the game arose from discussions between David Bulger and Dr. Dianne Cook as a first attempt to grow a 2-D game into higher dimensions. It has been further developed by Spencer Bradley, Dr. Hadley Wickham , Dr. Heike Hofmann and myself. 
% 
% The goal of the game is to connect a square in a 4-D cube with all four corners painted with your color. The game is a combination of ``Connect 4'', ``Tic Tac Toe'', and ``Dots and Lines'', only it is played in 4-D.  It is similar to `Connect 4' in that you need all four dots of your color, `Dots and Lines' in that you make a square, and the strategies from `Tic Tac Toe'. 

% The FaceOff \href{http://www.public.iastate.edu/~bigbear/FaceOff/index.html}{website} has a video and PDF tutorial.


\item{\bf IEEE Infovis Competition:} 2007 Infovis visualization competition
\begin{itemize}
  \item Internet Movie Data Base (IMDB) competition.
  \item Worked on the genres of the database using the ``Many Eyes'' web application.
  \item Helped edit and produce the \href{http://had.co.nz/infovis-2007/}{movie} that the working group submitted.
\end{itemize}

% The 2007 Infovis competition dealt with Internet Movie Data Base (IMDB). I contributed by doing work on the genres of IMDB's movie database through the use of Many Eyes.  The link for the movie that I helped edit is located on Dr. Hadley Wickham's \href{http://had.co.nz/infovis-2007/}{website}.

\end{itemize}

% \subsection{Honors Research}
% \begin{itemize}
%   \item Modernize a strain gauge and monitor into a LabView program.
%   \item Strain gauge was used to determine friction on the material.
%   \item Supervised by Dr. Sriram Sundararajan.
% \end{itemize}

% The Freshman Honors program provides a wonderful opportunity for freshman undergrad students to participate in research.  I participated under Dr. Sriram Sundararajan, a Mechanical Engineer professor,  on micro/nanoscale tribology.  The goal of the project was to take the old strain gauge and strain gauge monitor and turn it into a modern-day strain gauge system.  Another program member and I learned how to use LabVIEW and then used our knowledge to program a system to interpret the readings from the strain gauge attached to the system.  The program that we made allowed the user to record the strain caused by the friction from the spinning material over time.  The program also calculated the coefficient of friction at each time.  We ran several test runs on the new system to check it.

\section{Work Experience}

  \subsection{\href{http://metamarkets.com/}{Metamarkets}}
  
    Software Engineer - Spring 2011 - Summer 2012
    
    \begin{tabular}{p{6.25cm} p{9.25cm}}
      {\bf Client-Side} & {\bf Server-Side} \\
      \hline
      \begin{itemize}
        \item Developed url framework
        \item Created the javascript foundation from scratch
        \item Utilized DVL and D3 to make visualizations and objects data dependant 
        \item Modularized the code to make it useable in multiple locations
        \item Sped up deployment of code and utilized browser caching
        \item Minimized calls to backend server to avoid excessive load
      \end{itemize} &
      \begin{itemize}
        \item Learned and implemented Node.js
        \item Developed / forked multiple public packages
        \item Developed user management system
        \item Developed / utilized user configuration file
        \item Allowed user configuration to work with self serve process
        \item Maintained client security
        \item Implemented custom server cluster to prevent request failure
        \item Merged multiple data sources per request
        \item Wrote own security check and error notification system
      \end{itemize}
    \end{tabular}
    
  % Metamarkets is a web-based platform that allows its clients, digital publishers from around the globe, to gain better understanding of trends in the billions of ads they sell daily. Metamarkets was created to provide large-scale data analytics services. The analytics range from an hourly and daily aggregation of basic information, to predictions of how many advertisement impressions should be sold. By grouping this service in one location, many companies can interface with Metamarkets without trying to produce a similar monolithic service from scratch.
  % At Metamarkets, I design the client facing servers to route secure data and a platform for the web browser for client-side interaction. Clients request their data and it is massaged into a small and usable state so the browser is able to view the data easily. Metamarkets’ servers power the data manipulation so that the client only needs a web browser. Once the web browser has a handle on the functional data, I use the latest web graphics, D3, to produce interactive visualizations. D3 (Data-Driven Documents) is a JavaScript library created to manipulate website documents based on data. Using D3, I help build our own internal tool called DVL (Dynamic Visualization Library). DVL allows the frontend team to think in terms of pipelines rather than constantly knowing every component’s state. This allows each step of the interaction process to be independent from one another, providing simpler code and smoother interactions. Using a medium that can leverage the latest technology facilitates a symbiotic relationship between the user and company.


\section{Open Source}
  
  \subsection{Node.js / Javascript}
  
    \begin{itemize}
      \item {\bf \href{http://github.com/schloerke/yalog}{yalog}} Yet Another Logger - 2012
        \begin{itemize}
          \item Request logger to allow for personalized logging messages within different log levels
          \item Barret Schloerke
        \end{itemize}
    \end{itemize}
  
    \begin{itemize}
      \item {\bf \href{http://github.com/vogievetsky/DVL}{DVL}} Data Visualization Legos - 2011-2012
        \begin{itemize}
          \item Functionally reactive javascript library to bind data to functions and visualizations
          \item Vadim Ogievetsky, Barret Schloerke
        \end{itemize}
    \end{itemize}

    \subsection{R Packages}
    \begin{itemize}
      \item{\bf \href{http://github.com/ggobi/cranvas/}{cranvas}} New Plotting Canvas for R - 2010
        \begin{itemize}
          \item New plotting canvas for R.  Produces more than a million points in less than a second.
          \item Dr. Wickham, Dr. Lawrence, Dr. Hoffman, Dr. Cook, Yihui Xie, Tengfei Yin, Marie Vendettuoli, Barret Schloerke
          \item Iowa State University
        \end{itemize}
      % \bibentry{r_cranvas}
      
      \item{\bf \href{http://github.com/hadley/helpr/}{helpr}} Help for R - Summer 2010
        \begin{itemize}
          \item Betters Friendly HTML documentation with links to other packages, function aliases, and function sources.  Finding information may be done with the comprehensive search bar.
          \item Barret Schloerke, Dr. Wickham
          \item Rice University and Revolution Analytics
        \end{itemize}
      % \bibentry{r_helpr}
    
      % Helpr is an R package that betters friendly HTML documentation. With links to other packages, function aliases, and function sources, finding information is a click away. Using the comprehensive search bar, searching across all R packages is quick and effortless.
      % 
      % The heart of helpr is hosted locally. No internet is required to display all of the documentation, while functionality of the search bar, RSS feed, and comment system requires an internet connection.
      % 
      % The development of helpr was made possible with generous support from Revolution Analytics and direction from Hadley Wickham.
      
      \item{\bf \href{http://github.com/ggobi/ggally/}{GGally}} ggplot2's Best Friend - Spring 2010
        \begin{itemize}
          \item Re-designed the scatterplot matrix to handle other variable types.
          \item Barret Schloerke, Dr. Wickham, Dr. Cook, Dr. Hofmann
          \item Iowa State University
        \end{itemize}
      
      
      % \bibentry{r_ggally}
      
      % Scatterplot matrices are limited to viewing multivariate data because they only allow pairwise plots of real-valued variables. We've re-designed the scatterplot matrix to handle other variable types, and implemented this in a new R package called Gally. GGally accepts common X vs. Y combinations of data types and makes an intelligent guess at the at the best type of pairwise plot to use, for example, mosaic plot, side-by-side boxplots, or scatterplot. The upper triangle, and lower triangle, contain different plot types for the same pair of variables which may be specified. Each sub plot may be retrieved, alterd, and/or completely replaced by a custom-produced display. The software is built upon Hadley Wickham's ggplot2 package, and builds on ideas contained in Jay Emersons gpairs plot in the YaleToolkit package. Future work will expand the offerings of composite types of displays.
      % 
      % Direction from Hadley Wickham (Rice), Dianne Cook (ISU), Heike Hofmann (ISU)
      \item{\bf \href{http://github.com/ggobi/DescribeDisplay}{DescribeDisplay}} Reproduce GGobi images in R - Fall 2009
        \begin{itemize}
          \item Produce publication quality graphics from GGobi's describe display plugin.
          \item Barret Schloerke
          \item Iowa State University
        \end{itemize}
      
      % \bibentry{r_describedisplay}
      % Produce publication quality graphics from output of GGobi's describe display plugin
      
      \item{\bf \href{http://github.com/ggobi/tourr}{tourr}} Geodesic interpolation and basis generation functions - Fall 2009
        \begin{itemize}
          \item Implements geodesic interpolation and basis generation functions that allow you to create new tour methods in R.
          \item Dr. Wickham, Dr. Cook, Barret Schloerke
          \item Iowa State University and Rice University
        \end{itemize}
      
      % \bibentry{r_tourr} 
      % Implements geodesic interpolation and basis generation functions that allow you to create new tour methods from R.
      
      \item{\bf \href{http://streaming.stat.iastate.edu/~dicook/geometric-data/}{GeoZoo}} A library of high dimension geometric objects - 2007
        \begin{itemize}
          \item Functions to produce cubes, spheres, simplices, polyhedra, polytops, tori, and mobius-like objects. Contains a new addition to the high dimension polytopes, a Hyper Ring Torus
          \item Barret Schloerke, Dr. Cook
          \item Iowa State University
        \end{itemize}
      
      % \bibentry{r_geozoo}
    
      %  \href{http://streaming.stat.iastate.edu/~dicook/geometric-data/}{website} was created to help the viewer's eyes calibrate to high dimensions. GeoZoo contains many data sets of high dimensional geometric shapes. This includes cubes, spheres, simplices, polyhedra, polytopes, tori, and mobius like objects. GeoZoo contains a new addition to the high dimension polytopes, a Hyper Ring Torus. This work was jointly supervised by Dr. Hadley Wickham and Dr. Dianne Cook.
      % 
      % From my work on GeoZoo, I submitted a paper on the objects to The R Journal.
      
    \end{itemize}
    
  \subsection{Interface}
    \begin{itemize}
      \item{\bf R-Touch:}  R on the iPod Touch / iPhone - 2009
      \begin{itemize}
        \item Application allows the user to perform on-the-fly coding, produce images, download data from the Internet, restore old code files, and save current executed code.  The coding session may then be viewed on a computer using the session ID.
        \item Barret Schloerke, Chris Kielion
        \item Iowa State University
      \end{itemize}
      % \bibentry{r_rtouch}
      
      % With new advances in developer technologies, putting the R programming language on the iPhone became possible.  Working with Chris Kielion, we have made the application a reality.  The application allows the user to perform on-the-fly coding, produce images, download data from the Internet, restore old code files, and save current executed code.  The coding session may then be viewed on a computer using the session ID.
    
    \end{itemize}



% \section{Computer Languages}
% \begin{itemize}
%   \item{\bf Expert} R, Javascript, HTML, MYSQL
%   \item{\bf Proficient} Java, C, CSS, \LaTeX{}, Objective-C

%   \item{\bf R:} Acquainted in Summer of 2007. 
%   \item{\bf HTML and CSS:} Learned in Summer of 2007. Used in my professional and personal work
%   \item{\bf \LaTeX{}:} Use it to create professional documents since summer of 2007
%   \item{\bf Objective-C:} Learned in Fall 2009. Class taken in Fall 2010.  Used in research projects
%   \item{\bf Java:} Learned in Fall of 2008. Used in professional projects
%   \item{\bf Javascript and PHP:} Learned in Fall of 2009.  Used in professional projects since then
% \item{\bf MYSQL:}  Learned in Fall of 2008.  Used along side of PHP
%   \item{\bf C:}  Learned and used to create robot functionality in Spring 2009 and used in Spring 2010 for Operating Systems class (Linux Shell and Threads)
% \item{\bf MIPS Assembly and VHDL:} Learned and programed a single-cycle and multi-cycle processor
% \item{\bf VBA:} Used in first semester of college
% \end{itemize}


\section{Websites}

\begin{itemize}
  
  \item{\bf Checkin System} Implemented a database management system for Iowa State's \href{http://www.greekweek.barretschloerke.com/}{Greek Olympics} (2010 - 2012) and \href{http://www.homecoming.barretschloerke.com/}{Iowa State's Homecoming Blood Drive} (2012) to keep tabs on participants.
  
  \item{\bf Helpr} \href{http://github.com/hadley/helpr}{Helpr} {R} package is written using \href{http://github.com/hadley/sinartra}{sinartra} which allows for embedding R in HTML when the page is generated. 2010
  
  \item{\bf Senior Design} 
  \href{http://seniord.ece.iastate.edu/dec1009/}{Senior Design} project description with links to meeting minutes and team documents. 2010
  
  \item{\bf Election Maps} \href{http://barretschloerke.com/Election/index.html}{Election Maps} includes two sections: Cartogram of the USA and Electoral Vote Towers. 2009
  
  \item{\bf Novartis Toolbox} \href{http://barretschloerke.com/Novartis/}{Toolbox} designed for generating standard statistical graphics in R. The mission of the website is to give the user the tools to learn basic and advanced Lattice skills through examples. 2008
  
  \item{\bf GeoZoo} \href{http://streaming.stat.iastate.edu/~dicook/geometric-data/}{GeoZoo} was created to help the viewer's eyes calibrate to high dimensions. GeoZoo contains many data sets of high dimensional geometric shapes. 2007
  
  \item{\bf FaceOff} \href{http://www.public.iastate.edu/~bigbear/FaceOff/index.html}{FaceOff}'s video and PDF tutorial. 2007 
  
\end{itemize}





\section{Undergraduate Service}
\subsection{Office / Leadership Positions}

\begin{itemize}

\item{\bf Iowa State University Hip Hop Club (Dub H):} Technology Chair and Executive Committee - 2007 to 2010
\begin{itemize}
  \item Made club self sustaining
  \item Plan, execute, and review all events
  \item Produce videos and DVDs for club
\end{itemize}
% Dub H is one of the largest Iowa State University campus organizations with over 450 members. In Dub H, I have been on the executive committee as the Technology Director since Spring semester of 2008. In the Fall of 2007, I helped make the club self-sustaining in that the club now owns its own camera, camera accessories, and editing software and equipment. As technology updates, I recommend software and hardware updates that deem necessary or beneficial. In the Fall of 2009, I updated the website and currently maintain it. The \href{http://www.stuorg.iastate.edu/dubh/}{website} now offers a google calendar and RSS feed to keep the members up-to-date. It also displays our public performances hosted by YouTube. The club performs in front of over 3000+ people within the course of a year. Being on the exec (of six people) board, we prepare for the multiple events and make certain they run smoothly.

\item{\bf GRIP Mentoring:} Leader/Friend - Fall 2007 to Spring 2010
\begin{itemize}
  \item Met with and provided guidance for a middle school student in Gilbert, IA
\end{itemize}
% Great Relationships In Pairs is a mentor program within Ames Youth and Shelter Services. I have met with the same Gilbert Middle School student for the past two and a half years. Each week we get together and I provide guidance on how to solve daily social problems as well as a fun hour for his week. It has been rewarding for both of us.

\item{\bf Greek Week Check-in:} Electronic Check-in for Events Spring 2010
\begin{itemize}
  \item Managed a system with more than 2300 records for 10+ events
  \item Linked to participants ISU card
\end{itemize}
% During Greek Week, over 2300 people need to check-in to 10+ events. Using the participants' ISU ID card, they can swipe their card to sign in. This process was manually done by three people and now it can be done with one person manning a computer. The program checks for the person in the database, whether or not they have filled out a waiver, and the participation given by each pairing. Security is used to make sure that the system cannot be altered. I created this program after gaining knowledge from a previous class project.

\item{\bf Eta Kappa Nu:}
\begin{itemize}
  \item Organized event to inform middle school students what a Computer Engineer does at Iowa State University - Service Chair - Spring 2010
  \item Oversaw the meetings and made certain that the events were run properly - Vice President - Spring 2009
\end{itemize}
% I helped oversee the meetings and made certain that the events were run properly.

\item{\bf FarmHouse Fraternity:} Scholarship Chair - Fall 2007 to Spring 2009.
\begin{itemize}
  \item Implemented five new programs to strengthen GPA
  \item Moved committee chair into an executive position
\end{itemize}
% Grades have always been important to me. After being appointed scholarship chair in the fall of 2007, I implemented over five new programs to strengthen the house's individual and group GPA. In the evenings, I was available to others in assisting them with their studies. My last semester term, I proposed that FarmHouse make this duty an executive position. The ISU chapter of FarmHouse voted to accept the position of Scholarship Chair to be on the executive committee. Subsequently, as a house we reached a new all time high GPA of  3.29.

\item{\bf E-Ball Subcommittee:} Committee Member - Spring/Fall 2007
\begin{itemize}
  \item Planned dance event for 800 people
  \item Specialized in food selection
\end{itemize}
% Engineers Week is a weeklong event at Iowa State University ending with the Engineers' Ball that hosts about 800 people. I was on the committee for the Engineers' Ball. While assisting in the planning of the overall event, my main job was to be in charge of food selection and quantity on a \$7000 budget.


\end{itemize}


\subsection{Annual Volunteer Activities}
\begin{itemize}

\item{\bf Donate Blood} - 3 hours - Fall 2007, Fall 2008, Fall 2009, Fall 2010
\item{\bf Production of ISU Hip Hop DVDs} - 50+ hours / semester - Spring 2007, Fall 2007, Spring 2008, Fall 2008, Spring 2009, Fall 2009
\item{\bf FarmHouse Service Projects in Which I Participated}
\begin{itemize} \itemsep 0in
\item{\bf Burritoville} - 3 hours/semster - Fall 2006 - 2010
\item{\bf Feed the Need} - 8 hours/semester - Spring 2007, Spring 2008
\item{\bf Highway Cleanup} - 2 hours/semester - Every Semester along Highway 30 (x7)
\item{\bf Forestry Cleanup} - 6 hours/semster - Spring 2008
\item{\bf Blue Sky Days} - 1 hour/semster - Fall 2006, Fall 2007, Fall 2008
\end{itemize}
\end{itemize}

\subsection{Participant in the Following}
\begin{itemize} \itemsep 0in
\item Student Volunteer at ``R'' conference held at Iowa State University
\item Lawn Display (finished 3rd)
\item ISU Statesmen Singers (ISU Mens Choir)
\item Yell Like Hell (group finished 2nd)
\item Varieties (Played in the band for a skit - 3 years)
\item Intramural athletic activities (racquetball, hockey, broomball, dodgeball, curling)
\item Iowa State University Honors Program
\item Origami Demonstration
\end{itemize}










\section{Classes}

{\footnotesize

  \begin{tabular}{ll}
  
    
    \begin{tabular}{p{1.5cm} p{3.5cm} p{.3cm} p{.3cm} p{0.75cm}}
      \multicolumn{4}{l}{Fall 2006} &\\
      \hline
      Chem 167 & Engnrs General Chem 167 & 4.0 & A- & \\
      Engr 101 & Engineering Orietn  & R & S &\\
      Engr 131 & Learning Comm Semnr & R & S &\\
      Engr 160 & Engr Prob w/Cpr Lab & 3.0 & A &\\
      Hon 121 & Freshman Honors Sem & 1.0 & S & Honors\\
      Lib 160 & Lib Instruction  & 0.5 & S & Honors\\
      Math 165 & AdvPl-Calculus I & 4.0 & T &\\
      Math 166 & Calculus II & 4.0 & A &\\
      Music 151B  & Statesmen Mens Choir & 1.0 & A &\\ 
      \\
    \end{tabular}
    &
    \begin{tabular}{p{1.5cm} p{3.5cm} p{.3cm} p{.3cm} p{0.75cm}}
      \multicolumn{5}{l}{Spring 2007}\\
      \hline
      Chem 167L & Enginr Gen Chem Lab & 1.0  & A- &\\
      Engr 170 & Engr Graphics and Desgn & 3.0 & B+ &\\
      Hon 290H & Special Prob Honors & 1.0 & S & Honors\\
      Hon 322F & Data: Graphical, Visual & 1.0 & S & Honors\\
      M E 102 & Mech Engr Orientatn & R & S &\\
      Math 265 & Calculus III & 4.0 & A &\\
      Music 151B & Statesmen Mens Choir & 1.0 & A &\\
      Phys 221H & Honors- Classic Phys I & 5.0 & A & Honors\\
      \\
      \\
    \end{tabular}
    \\
  
    \begin{tabular}{p{1.5cm} p{3.5cm} p{.3cm} p{.3cm} p{0.75cm}}
      \multicolumn{3}{l}{Summer 2007} & & \\
      \hline
      E M 274 & Statics of Engineer & 3.0 & A\\
      \\
    \end{tabular}
    & 
    \\
  
    \begin{tabular}{p{1.5cm} p{3.5cm} p{.3cm} p{.3cm} p{0.75cm}}
      \multicolumn{5}{l}{Fall 2007}\\
      \hline
      E M 324 & Mechan of Materials & 3.0 & &\\
      M E 202 & Mech Engr Pro Plan & R & &\\
      Math 267 & Diff Eq and Transfms & 4.0 & &\\
      Phys 222H & Honors- Classical Phys II & 5.0 & &Honors\\
      Stat 305 & Engineering Stat & 3.0 & & \\
      \\
    \end{tabular}
    &
    \begin{tabular}{p{1.5cm} p{3.5cm} p{.3cm} p{.3cm} p{0.75cm}}
      \multicolumn{5}{l}{Spring 2008}\\
      \hline
      E M  345 & Dynamics & 3.0 & A- &\\
      Econ 102H & Honors Prin MacroEc & 3.0 & B+ &Honors\\
      M E 231 & Engr Thermodynams I & 3.0 & P&\\
      M E 270 & Intr Mch Eng Design & 3.0 & P&\\
      MAT E 272 & Prin Matrls Sci\&Eng & 2.0 & A&\\
      \\
    \end{tabular}
    \\
  
    \begin{tabular}{p{1.5cm} p{3.5cm} p{.3cm} p{.3cm} p{0.75cm}}
      \multicolumn{5}{l}{Fall 2008}\\
      \hline
      Com S 227 & Intro to Objt Orien Prg & 4.0 & A &\\
      Cpr E 281 & Digital Logic & 4.0 & A- &\\
      EE 201 & Electrical Circuits & 4.0 & A &\\
      Psych 280 & Social Psychology & 3.0 & P &\\
      \\
      \\
    \end{tabular}
    &
    \begin{tabular}{p{1.5cm} p{3.5cm} p{.3cm} p{.3cm} p{0.75cm}}
      \multicolumn{5}{l}{Spring 2009}\\
      \hline
      Com S 228 & Intro to Data Structur & 3.0 & A- &\\
      Cpr E 166 & Pro Prog Orientatn & R & S &\\
      Cpr E 288 & Embedde Sys I:Intro & 4.0 & B+ &\\
      Cpr E 310 & Theoret Found Cpr E & 3.0 & A &\\
      W S 201 & Intro Women's Studi & 3.0 & A &\\
      \\
    \end{tabular}
    \\
    \begin{tabular}{p{1.5cm} p{3.5cm} p{.3cm} p{.3cm} p{0.75cm}}
      \multicolumn{5}{l}{Fall 2009}\\
      \hline
      Com S 309 & Sftwr Dev Practces & 3.0 & A- &\\
      Cpr E 381 & Cptr Org \& Asmb Prog & 4 & A &\\
      Engl 314 & Technical Communctn & 3.0 & B &\\
      I E 409 & Intdiscp Sys Effctv & 3.0 & A- &\\
      Relig 323 & Science \& Religion & 3.0 & B- &\\
      S E 319 & Sftwr Cnstr \& User In & 3.0 & A &\\
    \end{tabular}
    &
    \begin{tabular}{p{1.5cm} p{3.5cm} p{.3cm} p{.3cm} p{0.75cm}}
      \multicolumn{5}{l}{Spring 2010}\\
      \hline
      AESHM 342 & Aesthetc of Everydy & 3.0 & A &\\
      Com S 311 & Dsn \& Analysis Algrth & 3.0 & B+ &\\
      Cpr E 308 & Operat Sys: Prn \& Pra & 4 & A- &\\
      Cpr E 491 & Senior Design Proj \& Pro & 3.0 & B+ &\\
      Cpr E 494 & Portfolio Assessmnt \& Pro & R & S &\\
    \end{tabular}
    \\
  
    \\
  
    \begin{tabular}{p{1.5cm} p{3.5cm} p{.3cm} p{.3cm} p{0.75cm}}
      \multicolumn{3}{l}{Summer 2010} & & \\
      \hline
      E E 230 & ELECTRON CIRCTS \& SYS & 4.0 & A\\
      \\
    \end{tabular}
    & 
    \\
  
    \begin{tabular}{p{1.5cm} p{3.5cm} p{.3cm} p{.3cm} p{0.75cm}}
      \multicolumn{5}{l}{Fall 2010}\\
      \hline
      Cpr E 388X & Embed Sy I:Miccntrl & 4.0 & A  & \\
      Cpr E 492 & Sr Design Projct II & 2.0 & A- &\\
      Cpr E 530 & Adv Protcl \& Nt Secur & 3.0 & A &\\
      HD FS 276 & Human Sexuality & 3.0 & A &\\
      HRI 383 & Intro Wine,Beer,Spi & 2.0 & P &\\
      \\
      \\
    \end{tabular}
    &
    \\
    
    \begin{tabular}{p{1.5cm} p{3.5cm} p{.3cm} p{.3cm} p{0.75cm}}
      \multicolumn{5}{l}{Fall 2012}\\
      \hline
      Stat 431 & Overview of Math Stat &  &  &\\
      Stat 615 & Intro to Reg \& Comput  &  &  &\\
      Stat 621 & Applied TS \& Forecasting  &  &  &\\
      Stat 601 & Statistics Colloquium & & P/NP   &\\
      Stat 405 & Intro to Data Analysis & & TA   &\\
      \\
      \\
    \end{tabular}
    &
    \\
    
  \end{tabular}
} % end size change


\newpage
\section{Personal Contacts}

  \begin{itemize}
  
    \item Dr. Hadley Wickham  \href{mailto:h.wickham@gmail.com}{h.wickham@gmail.com}\\
      Duncan Hall 2058\\
      Rice University\\
      Houston, TX 77005

    \item Dr Dianne Cook.  \href{mailto:dicook@iastate.edu}{dicook@iastate.edu}\\
      325 Snedecor Hall, Department of Statistics \\
      Iowa State University

    \item Dr. Michael Driscoll. \href{mailto:mike@metamarkets.com}{mike@metamarkets.com}\\
      \href{http://www.metamarkets.com}{Metamarkets}\\
      San Francisco, CA

  \end{itemize}
  
% \bibliography{BarretCV}

\end{document}

